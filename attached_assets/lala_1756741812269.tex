\usepackage{babel}
\usepackage[utf8]{inputenc}
\usepackage{textcomp} % For degree symbol and other text symbols
\usepackage{xcolor}
\usepackage{etoolbox}
\usepackage{pagecolor}
\definecolor{antiquepaper}{RGB}{250,248,240}
%\pagecolor{antiquepaper}
\pagecolor{white}
% Only two colors: black and green
\definecolor{darkgreen}{RGB}{0,80,0}
\definecolor{black}{RGB}{0,0,0}
\definecolor{shadecolor}{rgb}{0.97,0.97,0.97}
\usepackage{graphicx}
\usepackage{wallpaper}
\usepackage{wrapfig,booktabs}
\usepackage{caption}
\captionsetup{skip=10pt}
\usepackage{fancyhdr}
\usepackage{lettrine}
\input Acorn.fd
\newcommand*\initfamily{\usefont{U}{Acorn}{xl}{n}}

%======================================================================
%   SECTION NUMBERING CONTROL - HIDE NUMBERS IN DOCUMENT TITLES
%======================================================================
% Keep section numbering for TOC but hide from section titles in document
\setcounter{secnumdepth}{3} % Number sections internally
\setcounter{tocdepth}{3}    % Show sections in TOC

% Ensure section counter starts from 1 (this is default but made explicit)
\setcounter{section}{0}

% Section numbers will appear in TOC but not above section titles in document
% This is controlled by the \titleformat command below

% Optional: If you want to manually reset section numbering at any point:
% \newcommand{\resetsectioncounter}{\setcounter{section}{0}}
%======================================================================

%======================================================================
%   TABLE OF CONTENTS STYLING
%======================================================================
\usepackage{tocloft}
\usepackage{titletoc}

% Style the "Table Of Contents" title - properly centered
\renewcommand{\cfttoctitlefont}{\centering\Huge\bfseries\color{darkgreen}}
\renewcommand{\cftaftertoctitle}{}
\setlength{\cftbeforetoctitleskip}{2cm}
\setlength{\cftaftertoctitleskip}{2cm}

% Override default contents name
\renewcommand{\contentsname}{Table Of Contents}

% Override the previous suppression
\makeatletter
\renewcommand\tableofcontents{%
    \if@twocolumn
      \@restonecoltrue\onecolumn
    \else
      \@restonecolfalse
    \fi
    \chapter*{\contentsname
        \@mkboth{%
           \MakeUppercase\contentsname}{\MakeUppercase\contentsname}}%
    \@starttoc{toc}%
    \if@restonecol\twocolumn\fi
    }

% Customize section entries in TOC - ensure numbering is visible
\renewcommand{\cftsecfont}{\color{darkgreen}\bfseries}
\renewcommand{\cftsecpagefont}{\color{darkgreen}\bfseries}
\renewcommand{\cftsecleader}{\cftdotfill{\cftdotsep}}
\setlength{\cftsecindent}{0em}
\setlength{\cftsecnumwidth}{2em} % Adequate space for section numbers

% Customize subsection entries in TOC
\renewcommand{\cftsubsecfont}{\color{black}}
\renewcommand{\cftsubsecpagefont}{\color{black}}
\setlength{\cftsubsecindent}{2em}
\setlength{\cftsubsecnumwidth}{2.5em}

% Add ornamental decoration before TOC
\newcommand{\tocornament}{\par\vspace{1em}\noindent\begin{minipage}{\textwidth}\noindent\textcolor{darkgreen}{\hrulefill~ \raisebox{-2.5pt}[10pt][10pt]{\leafright \decofourleft \decothreeleft  \aldineright \decotwo \floweroneleft \decoone   \floweroneright \decotwo \aldineleft\decothreeright \decofourright \leafleft} ~  \hrulefill}\end{minipage}\par\vspace{1em}}

% Simple TOC command without ornaments
\newcommand{\customtoc}{%
  \pagenumbering{gobble} % Disable page numbering completely
  \newpage
  \thispagestyle{empty}
  \tableofcontents
  \newpage
}
%======================================================================

%======================================================================
%   DROP CAP COMMAND (Bulletproof Version)
%======================================================================
% This version uses \lettrine for the big letter ONLY (note the empty {}),
% and then manually prints the rest of the word (#2) as normal text.
% This bypasses all small caps formatting.
\newcommand{\dropcap}[1]{\lettrine[lines=3]{\initfamily\color{darkgreen}{#1}}{}}
%======================================================================

\usepackage{geometry}
\geometry{
tmargin=3cm, 
bmargin=3cm, 
lmargin=3cm, 
rmargin=3cm,
headheight=1.5cm,
headsep=0.8cm,
footskip=0.5cm}

\renewcommand{\familydefault}{pplj} 
\setlength{\parskip}{1.3ex plus 0.2ex minus 0.2ex}
\usepackage{fourier-orns}

\newcommand{\ornamento}{\par\vspace{2em}\noindent\begin{minipage}{\textwidth}\noindent\textcolor{darkgreen}{\hrulefill~ \raisebox{-2.5pt}[10pt][10pt]{\leafright \decofourleft \decothreeleft  \aldineright \decotwo \floweroneleft \decoone   \floweroneright \decotwo \aldineleft\decothreeright \decofourright \leafleft} ~  \hrulefill}\end{minipage}\par\vspace{2em}}

\newcommand{\ornpar}{\noindent \textcolor{darkgreen}{ \raisebox{-1.9pt}[10pt][10pt]{\leafright} \hrulefill \raisebox{-1.9pt}[10pt][10pt]{\leafright \decofourleft \decothreeleft  \aldineright \decotwo \floweroneleft \decoone}}}
\newcommand{\ornimpar}{\textcolor{darkgreen}{\raisebox{-1.9pt}[10pt][10pt]{\decoone \floweroneright \decotwo \aldineleft \decothreeright \decofourright \leafleft} \hrulefill \raisebox{-1.9pt}[10pt][10pt]{\leafleft}}}

\makeatletter
\def\headrule{{\color{darkgreen}\raisebox{-2.1pt}[10pt][10pt]{\leafright} \hrulefill \raisebox{-2.1pt}[10pt][10pt]{~~~\decofourleft \decotwo\decofourright~~~} \hrulefill \raisebox{-2.1pt}[10pt][10pt]{ \leafleft}}}
\makeatother

\fancyhf{}
\usepackage{titlesec}
\titleformat{\section}[display]
{\normalfont\Large\bfseries\centering}
{}{0pt}{\Large}
\titlespacing*{\section}{0pt}{50pt}{40pt}

\newcommand{\sectionbreak}{\clearpage}
\renewcommand{\sectionmark}[1]{\markboth{#1}{#1}}
\newcommand{\estcab}[1]{\itshape\textcolor{black}{\nouppercase #1}}
\fancyhead[R]{\estcab{\leftmark}}
\fancyhead[L]{}
\fancyhead[C]{}

%======================================================================
%   ALTERNATING FOOTER ORNAMENTS FOR ODD/EVEN PAGES
%======================================================================
% Define ornaments for odd pages (original direction)
\newcommand{\ornodd}{\noindent \textcolor{darkgreen}{ \raisebox{-1.9pt}[10pt][10pt]{\leafright} \hrulefill \raisebox{-1.9pt}[10pt][10pt]{\leafright \decofourleft \decothreeleft  \aldineright \decotwo \floweroneleft \decoone}}}

% Define ornaments for even pages (reversed direction)
\newcommand{\orneven}{\textcolor{darkgreen}{\raisebox{-1.9pt}[10pt][10pt]{\decoone \floweroneright \decotwo \aldineleft \decothreeright \decofourright \leafleft} \hrulefill \raisebox{-1.9pt}[10pt][10pt]{\leafleft}}}

% Set up alternating footers
\fancyfoot[C]{%
  \ifodd\value{page}%
    \ornodd \\ \large \sffamily\bf \textcolor{darkgreen}{\leafNE ~~~ \thepage ~~~ \reflectbox{\leafNE}}%
  \else%
    \orneven \\ \large \sffamily\bf \textcolor{darkgreen}{\reflectbox{\leafNE} ~~~ \thepage ~~~ \leafNE}%
  \fi%
}
%======================================================================

%======================================================================
%   UNIFIED WRAPPING COMMANDS (ALWAYS WRAP RIGHT) - FIXED VERSION
%======================================================================

% Wraps an image on the right side of the page.
% Usage: \WrapFig[width]{image_filename}{caption}
\newcommand{\WrapFig}[3][0.28\textwidth]{%
  \setlength{\columnsep}{12pt}
  \setlength{\intextsep}{8pt}
  \begin{wrapfigure}{R}{#1}
    \centering
    \includegraphics[width=\linewidth]{#2}
    \caption{#3}
  \end{wrapfigure}
}

% Wraps a table on the right side of the page - OPTIMIZED FOR SMALLER WIDTHS
% Usage: \WrapTable[width]{caption}{column format}{table content}
\newcommand{\WrapTable}[4][8cm]{%
  \setlength{\columnsep}{15pt}
  \setlength{\intextsep}{12pt}
  \begin{wraptable}{R}{#1}
    \centering
    \scriptsize % Changed from \footnotesize to \scriptsize for better fit
    \caption{#2}
    \vspace{0.2cm}
    \begin{tabular}{#3}
      #4
    \end{tabular}
  \end{wraptable}
}
%======================================================================

%======================================================================
%   PAGE NUMBERING CONTROL SYSTEM
%======================================================================
% Boolean to track if we've started content
\newif\ifcontentstarted
\contentstartedfalse

% Redefine section to start numbering on first section
\let\oldsection\section
\renewcommand{\section}[1]{%
  \ifcontentstarted
    % Do nothing, content already started
  \else
    % First section - start page numbering
    \pagenumbering{arabic}%
    \setcounter{page}{1}%
    \pagestyle{fancy}%
    \contentstartedtrue
  \fi
  \oldsection{#1}%
}

% Start document with no page numbering
\AtBeginDocument{%
  \pagenumbering{gobble}%
  \pagestyle{empty}%
}
%======================================================================

\usepackage{lipsum}
\setlength{\parindent}{1em}
\renewcommand{\footnoterule}{\vspace{-0.5em}\noindent\textcolor{darkgreen}{\decosix \raisebox{2.9pt}{\line(1,0){200}} \lefthand} \vspace{.5em} }
\usepackage[hang,splitrule]{footmisc}
\addtolength{\footskip}{0.5cm}
\setlength{\footnotemargin}{0.3cm}
\setlength{\footnotesep}{0.4cm} 
\usepackage{chngcntr}
\counterwithout{figure}{section}
\counterwithout{table}{section}
